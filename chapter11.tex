\documentclass{article}
\usepackage{assets/problem-set}

\begin{document}

\section*{Problem 11.1}
What is the change in entropy when $0.7~\unit{\meter\cubed}$ of
$\ch{CO2}$ and $0.3~\unit{\meter\cubed}$ of $\ch{N2}$, each at
$1~\unit{\bar}$ and $25~\unit{\degreeCelsius}$ blend to form a gas
mixture at the same conditions? Assume ideal gases.
\begin{solution}
  \begin{gather*}

    \intertext{Label $\ch{CO2}$ and $\ch{N2}$ as (1) and (2) respectively}
    V_1 = 0.7~\unit{\meter\cubed} \qquad V_2 = 0.3~\unit{\meter\cubed}

    \intertext{For ideal gases it follows that:}
    x_1 = 0.7 \qquad x_2 = 0.3 \\
    P = 1~\unit{\bar} \qquad T = 298.15~\unit{\kelvin} \\
    n = \frac{P\sum_{i}^{~}V_i}{RT} \\
    n = 40.340~\unit{\mole} \\
    \Delta S = -nR\sum_{i}^{~}x_i\ln x_i \\
    \boxed{\Delta S = 204.885~\unit{\joule\per\kelvin}}

  \end{gather*}
\end{solution}

\section*{Problem 11.2}
A vessel, divided into two parts by a partition, contains
$4~\unit{\mole}$ of nitrogen gas at $75~\unit{\degreeCelsius}$ and
$30~\unit{\bar}$ on one side and $2.5~\unit{\mole}$ of argon gas at
$130~\unit{\degreeCelsius}$ and $20~\unit{\bar}$ on the other. If the
partition is removed and the gases mix adiabatically and completely,
what is the change in entropy? Assume nitrogen to be an ideal gas
with $C_V=(5/2)R$ and argon to be an ideal gas with $C_V=(3/2)R$.
\begin{solution}
  \begin{gather*}

    \intertext{Label \ch{N2} and \ch{Ar} as (1) and (2) respectively}
    n_1 = 4~\unit{\mole} \qquad n_2 = 2.5~\unit{\mole} \\
    t_1 = 75~\unit{\degreeCelsius} \qquad t_2 = 130~\unit{\degreeCelsius} \\
    P_1 = 30~\unit{\bar} \qquad P_2 = 20~\unit{\bar} \\
    C_{V,\,1} = (5/2)R \qquad C_{V,\,2} = (3/2)R

    \intertext{Find $T$ after mixing by energy balance:}
    n_1 C_{V,\,1}(T-T_1) = n_2 C_{V,\,2} (T_2 - T)
    T = 365.125~\unit{\kelvin}

    \intertext{Find $P$ after mixing:}
    V = V_1 + V_2 \\
    \frac{(n_1 + n_2)RT}{P} = \frac{n_1RT_1}{P_1} + \frac{n_2RT_2}{P_2}
  \end{gather*}
\end{solution}

\end{document}
